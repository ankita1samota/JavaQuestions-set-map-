1.inheritance & polymorphism
2.encapsulation & abstraction
3.string and pear classes
4.exception handling
5.JUnit in java
//all topics to be include

6.arrays in java
7.collection in java
8.file handling in java
9.JDBC
10.github & loggers in java

//presentation

//questionaries(100 questions) & feedback

-error identification(30)/4
-output questions(30)/4
-scenerion based(30)/problem code/4(from hackerrank and leetcode
-MCQs(100)/25*



ERROR IDENTIFICATION:
=====================

1.
import java.util.HashSet;
import java.util.Set;

public class Test {
    public static void main(String[] args) {
        Set<String> fruits = new HashSet<>();
        fruits.add("Apple");
        fruits.add("Banana");
        fruits.add("Apple");
        System.out.println(fruits.get(0));
    }
}


2.import java.util.HashMap;
import java.util.Map;

public class Test {
    public static void main(String[] args) {
        Map<Integer, String> map = new HashMap<>();
        map.put(1, "One");
        map.put(1, "Two");
        System.out.println(map.get(1, "One"));
    }
}


3.
import java.util.Map;
import java.util.TreeMap;

public class Test {
    public static void main(String[] args) {
        Map<String, Integer> map = new TreeMap<>();
        map.put(null, 1);
        System.out.println(map);
    }
}


4.
import java.util.Set;
import java.util.HashSet;

public class Test {
    public static void main(String[] args) {
        Set<String> set = new HashSet<>();
        set.add(null);
        System.out.println(set);
    }
}


5.
import java.util.HashSet;
import java.util.Set;

public class Test {
    public static void main(String[] args) {
        Set<String> set = new HashSet<>();
        set.add("A");
        set.add("B");

        for (String s : set) {
            if (s.equals("B")) {
                set.remove(s);
            }
        }
        System.out.println(set);
    }
}


6.
import java.util.TreeSet;
import java.util.Set;

public class Test {
    public static void main(String[] args) {
        Set<String> set = new TreeSet<>();
        set.add(null);
        System.out.println(set);
    }
}


7.
import java.util.Arrays;
import java.util.List;
import java.util.HashSet;

public class Test {
    public static void main(String[] args) {
        List<String> list = Arrays.asList("A", "B");
        Set<String> set = new HashSet<>(list);
        set.add("C");
        list.add("D");
        System.out.println(set);
    }
}


8.
import java.util.HashMap;
import java.util.Map;

public class Test {
    public static void main(String[] args) {
        Map<Integer, String> map = new HashMap<>();
        map.put(1, "One");
        map.put(2, "Two");
        map.put(3, "Three");
        System.out.println(map);
    }
}


9.
import java.util.HashMap;
import java.util.Map;

public class Test {
    public static void main(String[] args) {
        Map<String, Integer> map = new HashMap<>();
        System.out.println(map.get("key"));
    }
}


10.
import java.util.Map;
import java.util.TreeMap;

public class Test {
    public static void main(String[] args) {
        Map<Integer, String> map = new TreeMap<>();
        map.put(3, "Three");
        map.put(1, "One");
        map.put(2, "Two");
        System.out.println(map);
    }
}


11.
import java.util.HashSet;
import java.util.Set;

public class Test {
    public static void main(String[] args) {
        Set set = new HashSet<>();
        set.add("A");
        set.add(1);
        System.out.println(set);
    }
}


12.
import java.util.HashMap;
import java.util.Map;

public class Test {
    public static void main(String[] args) {
        Map<Integer, String> map = new HashMap<>();
        map.put(1, "One");
        System.out.println(map.get("1"));
    }
}


13.
import java.util.HashSet;
import java.util.Set;

class Employee {
    int id;
    Employee(int id) { this.id = id; }

    @Override
    public int hashCode() { return id; }
}

public class Test {
    public static void main(String[] args) {
        Set<Employee> set = new HashSet<>();
        set.add(new Employee(1));
        set.add(new Employee(1));
        System.out.println(set.size());
    }
}


14.
import java.util.EnumMap;

enum Day { MON, TUE, WED }

public class Test {
    public static void main(String[] args) {
        EnumMap<Day, String> map = new EnumMap<>(Day.class);
        map.put(Day.MON, "Monday");
        map.put(null, "Holiday");
        System.out.println(map);
    }
}


15.
import java.util.HashMap;
import java.util.Map;

public class Test {
    public static void main(String[] args) {
        Map<Integer, String> map = new HashMap<>();
        map.put(1, "One");
        map.put(1, "New One");
        System.out.println(map.size());
    }
}











OUTPUT QUESTIONS:
================

1.What is the output of the following code?
import java.util.HashMap;
import java.util.Map;
public class Test {
public static void main(String[] args) 
{
 Map<Integer, String> map = new HashMap<>();
 map.put(101, "Red");
 map.put(103, "Green");
 map.put(102, "Yellow");
 
 Map<Integer,String> map2 = new HashMap<>();
 map2.put(115, "Brown");
 map2.put(120, "Purple");
 map.putAll(map2);
 System.out.println(map);
 }
}


2. What is the output of the following program?

import java.util.HashMap;
import java.util.Map;
public class Test {
public static void main(String[] args) 
{
 Map<Integer, String> map = new HashMap<>();
 map.put(101, "Red");
 map.put(103, "Green");
 map.put(102, "Yellow");
 map.remove(104);
 map.remove(106,"Pink");
 System.out.println(map);
 }


3.Is there any error in the program? If not, what is the output of program code?

import java.util.HashMap;
import java.util.Map;
public class Test {
public static void main(String[] args) 
{
 Map<String, String> map = new HashMap<>();
 map.put("A", "Apple");
 map.put("B", "Boy");
 map.put("C", "Cat");
 map.replace("C", "Element");
 map.remove("A","Apple");
 System.out.println(map);
 }
}

     


4. Is code will compile successfully? If yes, what will be the output of the following code?

import java.util.HashMap;
import java.util.Map;
public class Test {
public static void main(String[] args) 
{
 Map<String, String> map = new HashMap<>();
 map.put("V", "Violet");
 map.put("I", "Indigo");
 map.put("B", "Blue");
 map.put("G", "Green");
 map.put("Y", "Yellow");
 
 String value = map.get("B");
 System.out.println(value);
 boolean entryKey = map.containsKey("B");
 System.out.println(entryKey);
 boolean entryValue = map.containsValue("Brown");
 System.out.println(entryValue);
 }
}

	  
5.What will be the output of the following Java program?

 import java.util.*;
    class Maps 
    {
        public static void main(String args[]) 
        {
            HashMap obj = new HashMap();
            obj.put("A", new Integer(1));
            obj.put("B", new Integer(2));
            obj.put("C", new Integer(3));
            System.out.println(obj);
        }
    }
	
a) {A 1, B 1, C 1}
b) {A, B, C}
c) {A-1, B-1, C-1}


6. What will be the output of the following Java program?

  import java.util.*;
    class Maps 
    {
        public static void main(String args[]) 
        {
            HashMap obj = new HashMap();
            obj.put("A", new Integer(1));
            obj.put("B", new Integer(2));
            obj.put("C", new Integer(3));
            System.out.println(obj.keySet());
        }
    }
a) [A, B, C]
b) {A, B, C}
c) {1, 2, 3}
d) [1, 2, 3]

7. What will be the output of the following Java program?

    import java.util.*;
    class Maps 
    {
        public static void main(String args[]) 
        {
            HashMap obj = new HashMap();
            obj.put("A", new Integer(1));
            obj.put("B", new Integer(2));
            obj.put("C", new Integer(3));
            System.out.println(obj.get("B"));
        }
    }
a) 1
b) 2
c) 3
d) null

8.What will be the output of the following Java program?

    import java.util.*;
    class Maps 
    {
        public static void main(String args[]) 
        {
            TreeMap obj = new TreeMap();
            obj.put("A", new Integer(1));
            obj.put("B", new Integer(2));
            obj.put("C", new Integer(3));
            System.out.println(obj.entrySet());
        }
    }
a) [A, B, C]
b) [1, 2, 3]
c) {A=1, B=2, C=3}
d) [A=1, B=2, C=3]

9.
// Java Program Implementing HashMap
import java.util.HashMap;
import java.util.Map;

public class MapCreationExample {
    
      public static void main(String[] args) 
    {
      
        // Create a Map using HashMap
        Map<String, Integer> map = new HashMap<>();
                
        // Displaying the Map
        System.out.println("Map elements: " + map);
    }
}

10.// Java Program to Demonstrate
// Working of Map interface

// Importing required classes
import java.util.*;

// Main class
class GFG {

    // Main driver method
    public static void main(String args[])
    {
        // Creating an empty HashMap
        Map<String, Integer> hm
            = new HashMap<String, Integer>();

        // Inserting pairs in above Map
        // using put() method
        hm.put("a", new Integer(100));
        hm.put("b", new Integer(200));
        hm.put("c", new Integer(300));
        hm.put("d", new Integer(400));

        // Traversing through Map using for-each loop
        for (Map.Entry<String, Integer> me :
             hm.entrySet()) {

            // Printing keys
            System.out.print(me.getKey() + ":");
            System.out.println(me.getValue());
        }
    }
}


11.
import java.util.HashMap;
import java.util.Map;
 
/** Copyright (c), AnkitMittal JavaMadeSoEasy.com */
public class MyClass {
    public static void main(String args[]) {
           Map<String, String> hashMap = new HashMap<String, String>();
           hashMap.put(new String("a"), "audi");
           hashMap.put(new String("a"), "ferrari");
           System.out.println(hashMap);
    }
 
}


12.
import java.util.IdentityHashMap;
import java.util.Map;
 
/** Copyright (c), AnkitMittal JavaMadeSoEasy.com */
public class MyClass {
    public static void main(String args[]) {
           Map<String, String> identityHashMap = new IdentityHashMap<String, String>();
           identityHashMap.put(new String("a"), "audi");
           identityHashMap.put(new String("a"), "ferrari");
           System.out.println(identityHashMap);
    }
}


13.
import java.util.Map;
import java.util.TreeMap;
 
/** Copyright (c), AnkitMittal JavaMadeSoEasy.com */
public class TreeMapTest {
    public static void main(String args[]) {
 
           Map<Integer, String> m = new TreeMap<Integer, String>();
           m.put(11, "audi");
           m.put(null, null);
           m.put(11, "bmw");
           m.put(null, "fer");
 
           System.out.println(m.size());
           System.out.println(m);
    }
 
}


14.
import java.util.Arrays;
import java.util.Comparator;
 
/** Copyright (c), AnkitMittal JavaMadeSoEasy.com */
public class MyClass {
    public static void main(String[] args) {
           String[] ar = { "c", "d", "b", "a", "e" };
           NestedClass in = new NestedClass();
           Arrays.sort(ar, in);
           for (String str : ar)
                  System.out.print(str + " ");
           System.out.println(Arrays.binarySearch(ar, "b"));
    }
 
    static class NestedClass implements Comparator<String> {
           public int compare(String s1, String s2) {
                  return s2.compareTo(s1);
           }
    }
}


15.
import java.util.LinkedHashSet;
import java.util.Set;
 
/** Copyright (c), AnkitMittal JavaMadeSoEasy.com */
public class LinkedHashSetTest {
    public static void main(String args[]) {
 
           Set s = new LinkedHashSet();
           s.add("1");
           s.add(1);
           s.add(3);
           s.add(2);
           System.out.println(s);
 
    }
}





SCENARIO BASED QUESTIONS
=========================

1: Caching
You need to load stock exchange security codes with price from a database and cache them for performance. The security codes need to be
refreshed say every 30 minutes. This cached data needs to be populated and refreshed by a single writer thread and read by several reader
threads. How will you ensure that your read/write solution is scalable and thread safe?


2.
Detect Duplicate Entries in a List
Scenario:
You are given a list of student names. Some students have mistakenly registered multiple times. Write a Java program using a Set to detect 
and display the duplicate student names.
Hint: Use a HashSet to track seen names and identify duplicates.

3.
Maps-STL: You are appointed as the assistant to a teacher in a school and she is correcting the answer sheets of the students. Each student can 
have multiple answer sheets. The teacher has queries to add marks, erase marks, and print marks for students. 


4.
Day 8: Dictionaries and Maps: You are given a phone book consisting of people's names and their phone numbers. After that, you will be given some 
person's name as a query. For each query, print the phone number of that person. 

5.
Sets-STL: You are given queries to add elements to a set, delete elements from a set, and check if an element is present in the set. 

6.
Set Mutations: You are given two sets and a series of mutations to perform on the first set, including union, intersection, difference,
 and symmetric difference operations. 

7.
Two Out of Three
Problem Statement:
Given three integer arrays nums1, nums2, and nums3, return a list of all integers that appear at least once in two of the three arrays.
The result should be sorted in ascending order.
Example:
java
Copy
Edit
Input: nums1 = [1,1,3,2], nums2 = [2,3], nums3 = [3]
Output: [3]
Solution Approach:
Utilize a Set to store unique elements from each array. Then, iterate through the sets to find elements that appear in at least two sets.


8.
Design HashMap
Problem Statement:
Design a HashMap without using any built-in hash table libraries. Implement the MyHashMap class with the following methods:
put(int key, int value): Inserts a (key, value) pair into the HashMap. If the key already exists, update the value.
get(int key): Returns the value to which the specified key is mapped, or -1 if the key does not exist.
remove(int key): Removes the key and its corresponding value if the key exists.
Solution Approach:
Implement the HashMap using an array of linked lists to handle collisions. Each index in the array represents a bucket, and each bucket contains a 
list of key-value pairs.


9.
Intersection of Two Arrays
Problem Statement:
Given two integer arrays nums1 and nums2, return an array of their intersection. Each element in the result must appear as many times as it shows 
in both arrays. The result can be returned in any order.
Example:
java
Copy
Edit
Input: nums1 = [1,2,2,1], nums2 = [2,2]
Output: [2,2]
Solution Approach:
Use a Map to count the occurrences of each element in the first array. Then, iterate through the second array, and for each element, check if it 
exists in the map with a non-zero count. If it does, add it to the result and decrement its count in the map.


10.
Find and Replace Pattern
Problem Statement:
Given a list of strings words and a string pattern, return a list of words[i] that match pattern. A word matches the pattern if there exists a 
bijection between every letter in pattern and every letter in the word.
Example:
java
Copy
Edit
Input: words = ["abc","deq","mee","aqq","dkd","ccc"], pattern = "abb"
Output: ["mee","aqq"]
Solution Approach:
Create a helper function to check if a word matches the pattern by mapping each character in the pattern to a character in the word using a Map.












MCQS
====

1.Which of these object stores association between keys and values?
a) Hash table
b) Map
c) Array
d) String

2.Which of these classes provide implementation of map interface?
a) ArrayList
b) HashMap
c) LinkedList
d) DynamicList

3.Which of these method is used to remove all keys/values pair from the invoking map?
a) delete()
b) remove()
c) clear()
d) removeAll()

4.Which of these method Map class is used to obtain an element in the map having specified key?
a) search()
b) get()
c) set()
d) look()

5.Which of these methods can be used to obtain set of all keys in a map?
a) getAll()
b) getKeys()
c) keyall()
d) keySet()

6.Which of these method is used to add an element and corresponding key to a map?
a) put()
b) set()
c) redo()
d) add()

7.Which of these classes implements Set interface?
a) ArrayList
b) HashSet
c) LinkedList
d) DynamicList

8.Which of these method of HashSet class is used to add elements to its object?
a) add()
b) Add()
c) addFirst()
d) insert()

9.Which collection allows you to retrieve elements in the order they were inserted?
a) ArrayList
b) LinkedList
c) LinkedHashMap
d) LinkedHashSet

10.Which collection class provides a way to store elements in key-value pairs with no duplicate keys?
a) ArrayList
b) Hashtable
c) TreeMap
d) LinkedHashMap

11.Which one of the following collection classes can be used to store unique & sorted objects?
a) HashSet
b) TreeSet
c) LinkedHashMap
d) ArrayList

12.Which collection class allows duplicate elements and maintains their insertion order?
a) HashSet
b) TreeSet
c) LinkedHashSet
d) HashMap

13.Which method is used to check if a specific element is present in a Set?
a) containsElement()
b) containsKey()
c) containsValue()
d) contains()

14.Which collection class in Java stores objects in the form of key-value pairs and maintains the order in which they were inserted?
a) Hashtable
b) LinkedHashMap
c) TreeMap
d) HashSet

15.Which collection class is synchronized and thread-safe?
a) ArrayList
b) HashSet
c) ConcurrentHashMap
d) LinkedList

16.Which collection class in Java provides a fixed-size collection that does not allow adding or removing elements?
a) ArrayList
b) LinkedList
c) HashSet
d) Arrays

17.What is the main difference between HashSet and TreeSet in Java?
a) HashSet allows duplicate elements, while TreeSet does not.
b) HashSet maintains insertion order, while TreeSet does not.
c) HashSet is unsorted, while TreeSet is sorted.
d) TreeSet allows null values, while HashSet does not.

18.Which collection class in Java does not allow null values?
a) HashSet
b) ArrayList
c) HashMap
d) TreeSet

19.Which collection class in Java is best suited for frequent search operations on a large collection?
a) HashSet
b) ArrayList
c) LinkedList
d) TreeMap

20.Which Map class must be preferred in multi-threading environment to maintain natural order of keys?
a) ConcurrentHashMap
b) ConcurrentSkipListMap
c) ConcurrentMap
d) all

21.TreeMap implements?
a) Dictionary
b) HashMap
c) AbstractMap
d) NavigableMap

22.Which Set class must be preferred in multi-threading environment, considering performance constraint?
a) HashSet
b) ConcurrentSkipListSet
c) LinkedHashSet
d) CopyOnWriteArraySet

23.Which Map class must be preferred in multi-threading environment, considering performance constraint?
a) Hashtable
b) CopyOnWriteMap
c) ConcurrentHashMap
d) ConcurrentMap

24.Which of these is synchronized?
a) TreeMap
b) HashMap
c) Hashtable
d) All

25.Which is the best performance Map implementation?
a) TreeMap
b) HashMap
c) LinkedHashMap
d) All are equal

26.Which of these is a thread-safe variant of ArrayList?
a) Vector
b) CopyOnWriteArrayList
c) LinkedList
d) HashSet

27.Which Java collection allows only unique elements and provides constant-time performance for basic operations?
a) ArrayList
b) HashSet
c) LinkedList
d) HashMap

28.Which of these methods is used to compare two sets for equality?
a) equals()
b) compareTo()
c) isEqual()
d) compare()

29.Which data structure is used internally by a HashMap for collision resolution in Java 8?
a) LinkedList
b) TreeMap
c) Binary Tree
d) Red-Black Tree

30.Which of these collections implements a LIFO (Last-In-First-Out) structure?
a) Queue
b) LinkedList
c) Stack
d) PriorityQueue

31.What happens if you try to insert a duplicate key in a HashMap?
a) It throws an exception
b) The old value is replaced with the new one
c) The duplicate key is ignored
d) It creates a new entry

32.Which of these methods is used to obtain all keys in a Map?
a) getKeys()
b) keySet()
c) fetchKeys()
d) mapKeys()

33.Which of these is NOT a method in the Map interface?
a) get()
b) put()
c) add()
d) containsKey()

34.Which collection is best suited for implementing a priority queue?
a) ArrayList
b) LinkedList
c) PriorityQueue
d) HashSet

35.Which Map implementation provides a thread-safe and lock-free approach to retrieving data?
a) HashMap
b) Hashtable
c) ConcurrentHashMap
d) TreeMap

36.Which of these collections maintains elements in natural ordering?
a) HashSet
b) TreeSet
c) ArrayList
d) HashMap

37.Which of these collections does NOT allow null keys?
a) HashMap
b) TreeMap
c) LinkedHashMap
d) WeakHashMap

38.Which Map class stores keys and values in sorted order?
a) TreeMap
b) LinkedHashMap
c) HashMap
d) HashTable

39.Which Set class stores unique elements in sorted order?
a) HashSet
b) TreeSet
c) LinkedHashSet
d) PrioritySet

40.Which of these methods is used to remove an element from a Set?
a) delete()
b) remove()
c) discard()
d) pop()

41.Which of these methods removes all mappings from a Map?
a) clear()
b) remove()
c) delete()
d) empty()

42.Which method does the Map interface use to insert or update a key-value pair?
a) insert()
b) add()
c) put()
d) append()

43.Which of these collection types can be used when maintaining unique, sorted data without duplicates?
a) HashSet
b) TreeSet
c) ArrayList
d) PriorityQueue

44.Which interface does LinkedHashMap implement in addition to Map?
a) Set
b) List
c) NavigableMap
d) SortedMap

45.What is the maximum number of null elements that a TreeSet can store?
a) None
b) One
c) Two
d) Any number

46.Which class does not allow duplicates and maintains insertion order?
a) TreeSet
b) HashSet
c) LinkedHashSet
d) LinkedList

47.Which of these collections can store both key-value pairs and also maintains the order of insertion?
a) TreeMap
b) LinkedHashMap
c) HashMap
d) ArrayList

48.Which of these Map implementations is best for high-performance operations when frequent access and updates are required?
a) LinkedHashMap
b) HashMap
c) TreeMap
d) Hashtable

49.Which of the following is NOT a valid implementation of the Set interface in Java?
a) HashSet
b) LinkedHashSet
c) TreeSet
d) PrioritySet

50.What does the entrySet() method in a Map return?
a) A Set of keys
b) A Set of values
c) A Set of key-value pairs
d) A Map of key-value pairs